\chapter*{Acknowledgements (READY FOR FEEDBACK, PROOFREAD)\markboth{Acknowledgements}{Acknowledgements}}
%\addcontentsline{toc}{chapter}{\protect\numberline{}Acknowledgements}

First and foremost, I wish to thank Prof.~Dr.~Armin B.~Cremers for allowing me to write on this self-chosen topic, and for his encouragement and critical questions. A big kudos also to my thesis advisor Daniel Speicher for helping finally find this thesis topic, for his support and guidance, and for his seemingly infinite patience with me during this process.

I'd also like to thank Sebastian Bergmann (the author of the famous PHPUnit), Roman Saul and Christian Kuhn (from the TYPO3 CMS project) who provided feedback, a different perspective and critical questions.

Thanks also go to Melanie Kelter, who mercilessly pointed out my typos and crooked sentences.

I also thank my fellow team members Henning Pingel, Marcus Krause and Helmut Hummel (from the TYPO3 Security Team) who have taught me most of what I know about web application security now.

Thanks also go to my father, who never stopped believing that I would someday finish this thesis (and who kept pushing and nudging me all the time).

Last but not least, I would like to thank Nenad Jovanovic for creating Pixy and PhpParser in the first place, and on whose work this thesis has been built. The saying about ``standing on the shoulders of giants'' concerning Open-Source projects really is true.

\section*{Tools and Services}

During the creation of this thesis, numerous tools and online services were involved: Kubuntu Linux and occasionally Mac OS X as operating systems, Kile for editing the \LaTeX source code, Bib\TeX for managing literature references, pdf\TeX for compiling it into PDF, Okular for viewing the generated PDF, Acrobat Reader for viewing the PDF with the proofreading annotations, Kate and Krusader's internal editor for editing the occasional plaint-text file, Eclipse and the vastly superior IntelliJ IDEA Community Edition for developing in Java, PhpStorm for developing in PHP, GIT hosted at Github for version-controlling the project source code and the \LaTeX sources, OmniGraffle for creating the spiffy diagrams with gradients, PhpParser and Pixy for parsing PHP code and dumping the parse trees as DOT files, ZgrViewer for viewing the DOT files and for exporting them as SVG, Google Chrome for viewing the SVG files, librsvg for convering SVG files into PDF, iWork Numbers for creating the bar chart, Mendeley for managing the literature PDFs, and Trello for project management.

Most of these tools are open-source projects, and without these tools, this thesis would not have been possible.
