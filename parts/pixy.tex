\chapter{The PHP Security Scanner Pixy (WORK IN PROGRESS)}
\label{pixy}\index{Pixy}\index{PhpParser}
Pixy~\cite{pixy} was created 2006/2007 as part of a dissertation by Nenad Jovanovic~\cite{pixy-dissertation}. It uses interprocedural data-flow analysis and includes the dedicated PhpParser tool~\cite{phpparser}. Pixy's approach is documented in~\cite{pixy-short, pixy-long, pixy-technical, pixy-dissertation}.

Pixy is able to recognize sources, sinks and sanitation functions specific for each vulnerability type. However, in its 2007 version, it only recognized simple functions, not method calls on objects or static function calls for a class.

Pixy could currently only scan one file at a time (including its dependencies) and only scans functions that actually are executed. This means that it could not scan the code of a complete class if there was no caller.

Development of Pixy had ceased after 2007. However, one of the original authors of Pixy had agreed to hand over maintenance so Pixy can be officially continued.

\section{The Pixy Project on the Web}

The Pixy project (including the source code, wiki and issue tracker) currently resides on Github at \url{https://github.com/oliverklee/pixy}. The related PhpParser project is located at \url{https://github.com/oliverklee/phpparser}.\index{Github}

\section{Technical Details}

As shown in figure~\ref{fig:pixy-data-structures} on page~\pageref{fig:pixy-data-structures}, Pixy uses a several-steps approach between the raw source code and the final data flow analysis. It makes use of the (modified) external libraries JFlex and CUP (and a Lex syntax definition file for PHP) to create the abstract syntax tree.\index{JFlex}\index{CUP}

\begin{figure}[htb]
 \includegraphics[scale=0.8]{images/Pixy-Arbeitsweise}
 \caption{The main data structures in Pixy}
 \label{fig:pixy-data-structures}
\end{figure}
