\chapter{The PHP Security Scanner Pixy}
\label{pixy}\index{Pixy}\index{PhpParser}
Pixy~\cite{pixy} was created in 2006/2007 as part of a dissertation by Nenad Jovanovic~\cite{pixy-dissertation}. It uses interprocedural data-flow analysis and includes the dedicated PhpParser tool~\cite{phpparser}. Pixy's approach is documented in~\cite{pixy-short, pixy-long, pixy-technical, pixy-dissertation}.

Pixy is able to recognize sources, sinks and sanitation functions specific for each vulnerability type. However, in its 2007 version, it only recognized simple functions, not method calls on objects or static function calls for a class.

Pixy could currently only scan one file at a time (including its dependencies), and it scans only functions that actually are executed. This means that it could not scan the code of a complete class if there was no caller.

Development of Pixy had ceased after 2007. However, one of the original authors of Pixy had agreed to hand over maintenance so Pixy can be officially continued.

\section{The Pixy Project on the Web}

The Pixy project (including the source code, wiki and issue tracker) currently resides on Github at \url{https://github.com/oliverklee/pixy}. The related PhpParser project is located at \url{https://github.com/oliverklee/phpparser}.\index{Github}

\section{Technical Details}

As shown in figure~\ref{fig:pixy-data-structures} on page~\pageref{fig:pixy-data-structures}, Pixy uses a several-steps approach between the raw source code and the final data flow analysis. It makes use of the (modified) external libraries JFlex and CUP (and a Lex syntax definition file for PHP) to create a PHP parse tree.\index{JFlex}\index{CUP}

\begin{figure}[htb]
 \begin{center}
   \includegraphics[scale=0.75]{images/Pixy-Arbeitsweise}
   \caption{From the PHP parse tree, Pixy generates a control-flow graph and P-TAC, using these for the dependency analysis and for the taint analysis.}
   \label{fig:pixy-data-structures}
 \end{center}
\end{figure}

\section{P-TAC as an Intermediate Representation in the Control-Flow Graph}
\label{p-tac}\index{P-TAC}

As an intermediate representation, Pixy uses a combination of a modified version of \emph{three-address code}\index{three-address code} (see section~\ref{tac}) called \emph{P-TAC} together with a control-flow graph (CFG).\index{control-flow grap} For addresses that contain data (variables, constants, literals), Pixy uses the general term \emph{place}.\index{place} Pixy represents \emph{places} in the class \texttt{AbstractPlace}.\index{AbstractPlace}

As described in~\cite{pixy-dissertation}, the main types of control-flow graph nodes are listed in table~\ref{table:p-tac} on page~\pageref{table:p-tac}.

\myTable{
\begin{tabular}{|l|p{5.5cm}|l|}
\hline
  \bb{CFG node} & \bb{description} & \bb{class} \\
\hline
 simple assignment  & $variable$ \texttt{=} $place$ & \texttt{AssignSimple} \\
\hline
 unary assignment  & $variable$ \texttt{=}  $operator\ place$ & \texttt{AssignUnary} \\
\hline
 binary assignment  & $variable$ \texttt{=} $place\ operator\ place$ & \texttt{AssignBinary} \\
\hline
 array assignment  & $variable$  \texttt{= array()} & \texttt{AssignArray} \\
\hline
 assignment by reference  & $variable$ \texttt{=\&} $variable$ & \texttt{AssignReference} \\
\hline
 unset & \texttt{unset($variable$)} & \texttt{Unset} \\
\hline
 global & \texttt{global} $variable$ & \texttt{Global} \\
\hline
 call preparation & a call node's predecessor & \texttt{CallPreparation} \\
\hline
 call & a function or method call & \texttt{Call} \\
\hline
 call return & a call node's successor & \texttt{CallReturn} \\
\hline
 basic block & a basic block containing other nodes & \texttt{BasicBlock} \\
\hline
 PHP function call & a call of one of PHP's built-in functions & \texttt{CallBuiltinFunction} \\
\hline
 unknown function call & a function call that cannot be resolved during analysis & \texttt{CallUnknownFunction} \\
\hline
 entry node & the entry node of the control-flow graph & \texttt{CfgEntry} \\
\hline
 exit node & an exit node of the control-flow graph & \texttt{CfgExit} \\
\hline
 constant definition & \texttt{define($key, value$)} & \texttt{Define} \\
\hline
 echo call & \texttt{echo $variable$} & \texttt{Echo} \\
\hline
 if & \texttt{if($true$|$false$) goto $target$} & \texttt{If} \\
\hline
 file inclusion & \texttt{include $variable$} & \texttt{Include} \\
\hline
\end{tabular}
}{The main types of control-flow graph nodes in P-TAC as used by Pixy. The class names are within the package \texttt{pixy.conversion.cfgnodes}.}{table:p-tac}

In Pixy, the class \texttt{TacConverter} is responsible for converting the PHP parse tree to P-TAC and a control-flow graph.\index{tacconverter}\label{tacconverter}
