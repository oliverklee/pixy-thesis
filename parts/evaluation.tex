\chapter{Experimental evaluation of the modified version of Pixy (WORK IN PROGRESS)}

In this section, we will look at the modified version of Pixy both with a code quality perspective as well as a functional perspective.

The original version of Pixy can be found at~\cite{pixy}, and the modified version of Pixy and the related PhpParser project are hosted at Github at \url{https://github.com/oliverklee/pixy} and at \url{https://github.com/oliverklee/phpparser}.\index{Github}

\section{Code quality}
\label{code-quality}
\index{code quality}
One of the aims of the thesis is to make Pixy a tool that is and will continue to be useful for other developers, both for using it and for contributing to the project. This includes that the Pixy's code is well-tested, well-readable and of general high quality. For measuring improvements in code quality, the author has decided to use three numbers that are relatively easy to measure:

\begin{itemize}
  \item the number of warnings and errors issued by javac 1.7 when run with the \texttt{-Xlint} option
  \item the number of warnings and errors issued by the PMD\footnote{``Project Mess Detector'', but there exists several explanations of what this acronym means~\cite{pmd-meaning}.}~\cite{pmd} source code analyzer for Java
    \index{PMD}
    \index{Project Mess Detector \see{PMD}}
  \item the number of JUnit unit tests and the number of failures and errors
  \index{unit test}
  \index{JUnit \see{unit tests}}
\end{itemize}

The aim of this thesis is to get the javac lint and PMD warnings as close to zero as possible and to get all unit tests to pass. In addition, all changes and new features should be covered with unit tests.

This only applies to the Pixy project as most of the code of the related PhpParser is generated, i.e., the author does not have much direct influence on the quality of that code.

\subsection{Java lint warnings}

Before the author made any changes, javac lint (version 1.7.0\_13) issued 688 warnings for Pixy (many of which may be due to Pixy originally being developed for Java 1.5).

\subsection{PMD}

The author chose a subset of the available Java-related PMD rule sets that fit to the scope of the Pixy project (e.\,g., a rule set for Android does not make sense for this non-Android project). Other rule sets were skipped as they provided too many false positives for this project (please see table~\ref{table:pmd-skipped} on page~\pageref{table:pmd-skipped} for a list).

The PMD version used for these tests was version 5.0.2 (the current version at the time of writing). To avoid changed numbers to do different behavior of subsequent versions, the PMD version was kept at 5.0.2 even if updates were available later.

A description of the rules included in the rule sets is provided in the PMD documentation~\cite{pmd-rulesets}.

Table~\ref{table:pmd-violations} on page~\pageref{table:junit-before} is a comparison of the number of PMD violations in the Pixy project before the author made any changes and after cleanup was finished.

\myTable{
\begin{tabular}{|l|l|r|r|}
\hline
 \bb{Rule set name} & \bb{rule set key} & \bb{before cleanup} & \bb{after cleanup} \\
\hline
 Basic & \texttt{java-basic} & 143 & \\
\hline
 Braces & \texttt{java-braces} & 358 & \\
\hline
 Clone Implementation & \texttt{java-clone} & 5 & \\
\hline
 Code Size & \texttt{java-codesize} & 262 & \\
\hline
 Coupling & \texttt{java-coupling} & 4809 & \\
\hline
 Design & \texttt{java-design} & 739 & \\
\hline
 Empty Code & \texttt{java-empty} & 41 & \\
\hline
 Finalizer & \texttt{java-finalizers} & 0 & \\
\hline
 Import Statements & \texttt{java-imports} & 23 & \\
\hline
 JUnit & \texttt{java-junit} & 274 & \\
\hline
 Migrations & \texttt{java-migrating} & 394 & \\
\hline
 Naming & \texttt{java-naming} & 1245 & \\
\hline
 Strict Exceptions & \texttt{java-strictexception} & 328 & \\
\hline
 String and StringBuffer & \texttt{java-strings} & 180 & \\
\hline
 Security Code Guidelines & \texttt{java-sunsecure} & 2 & \\
\hline
 Type Resolutions & \texttt{java-typeresolution} & 160 & \\
\hline
 Unnecessary & \texttt{java-unnecessary} & 75 & \\
\hline
 Unused Code & \texttt{java-unusedcode} & 24 & \\
\hline
 \bb{Total} &  & \bb{9086} & \bb{} \\
\hline
\end{tabular}
}{Number of PMD violations in the Pixy project before and after cleanup}{table:pmd-violations}

\myTable{
\begin{tabular}{|p{3.1cm}|p{3.1cm}|r|p{4.7cm}|}
\hline
 \bb{Rule set name} & \bb{rule set key} & \bb{violations } & \bb{reason for skipping} \\
\hline
 Android & \texttt{java-android} & 0 & n/a \\
\hline
 Comments & \texttt{java-comments} & 1829 & The ``line too long'' rule is too restrictive. \\
\hline
 Controversial & \texttt{java-controversial} & 2610 & The name says it all. :-)  \\
\hline
 J2EE & \texttt{java-j2ee} & 5 & n/a \\
\hline
 Java Beans & \texttt{java-javabeans} & 558 & n/a \\
\hline
 Jakarta Commons Logging & \texttt{java-logging-jakarta-commons} & 0 & n/a \\
\hline
 Java Logging & \texttt{java-logging-java} & 505 & \texttt{System.out.print} actually is okay for this application.\\
\hline
 Optimization & \texttt{java-optimizations} & 7880 & Too many low-priority ``\ldots could be declared final'' messages.\\
\hline
\end{tabular}
}{PMD rule sets that have been skipped}{table:pmd-skipped}

\paragraph{Note:} As PMD does not provide a count of violations when using the \texttt{text} output format on the command line, the output was piped through \texttt{wc -l} to count the number of violations.

\subsection{JUnit unit tests}

The numbers in table~\ref{table:junit-before} on page~\pageref{table:junit-before} show the state of Pixy before and after the modifications. The code coverage has been determined using the Cobertura tool.~\cite{cobertura}

\myTable{
\begin{tabular}{|l|r|r|}
\hline
  \bb{Metric} & \bb{before modification} & \bb{after modification} \\
\hline
  Number of executed tests & 363 & \\
\hline
  Test errors & 0 & \\
\hline
  Test failures & 1 & \\
\hline
  Line coverage & 67\,\% & \\
\hline
  Branch coverage & 44\,\% & \\
\hline
\end{tabular}
}{JUnit test results before the modifications, using Java 1.7 and JUnit 3}{table:junit-before}
