\chapter{Alias analysis}

When performing static code analysis, a good alias analysis is helpful as it can both increase recall and precision. A good recall is important as it will allow Pixy to find more vulnerabilities. A good recall is important to reduce noise, thus making the results more meaningful for the developers: If there are too many meaningless warnings, developers just tend to ignore them (or stop using the tool).~\cite{understanding-value}

For understanding the intricacies of alias analysis for PHP, it is important to first have a firm grip on the way references work in PHP (which is quite different from the way aliases work e.\,g., in C or Java). Thus a big part of the exiting work on alias analysis does not directly apply to PHP.~\cite[page~24]{pixy} Subsection~\ref{sec:references} on page~\pageref{sec:references} provides more information on this.

\section{Alias analysis in Pixy}

For its alias analysis, Pixy uses a modified version of the points-to-analysis described by Khedker et.\,al~\cite[page 119ff]{data-flow-analysis}, including the concept of ``must'' and ``may'' aliases.

\paragraph{Must-aliases} are relationships between variables that are aliases to the same ZVAL independent of the actual executed program path.

\paragraph{May-aliases} are relationships between variables that are aliases only for some executed program paths.

This separation helps in cases where two variables \texttt{\$a} and \texttt{\$b} are tainted and \texttt{\$a} gets sanitized. If \texttt{\$a} and \texttt{\$b} are must-aliases, \texttt{\$b} can safely considered to be sanitized as well. However, if both variables are may-aliases, the scanner should make a conservative decision and consider \texttt{\$b} still to be tainted.


\subsection{Intraprocedural alias analysis}

As described in \cite{pixy}, Pixy keeps record for all must-aliases and may-aliases for each line of program code. The must-aliases are represented as unordered and disjoint sets of variables that are certain to be references to the same ZVAL at a certain point at the program. May-aliases are represented the same way. Let's have a look at an example.

Note: In these examples, the sets of must-aliases and may-aliases always refer to point of execution after the last code line listed above.

At the beginning of a function or method, the sets of may-aliases and must-aliases is empty:

$mustAliases = \{\}, mayAliases = \{\}$

When a reference is created, the pair of both variables is added to the must-aliases:

\begin{phpcode}
$a = &$b;
\end{phpcode}
$mustAliases = \{(a, b)\}, mayAliases = \{\}$


If there is a branch condition, the aliases set within the branch still are considered to be must-aliases, but \emph{only within that particular branch}.

\begin{phpcode}
$a = &$b;
if (...) {
  $c = &$d;
\end{phpcode}
$mustAliases = \{(a, b), (c, d)\}, mayAliases = \{\}$

\begin{phpcode}
$a = &$b;
if (...) {
  $c = &$d;
  $e = &$d;
\end{phpcode}
$mustAliases = \{(a, b), (c, d, e)\}, mayAliases = \{\}$

Now, after the branch, the scanner needs to change the must-aliases that have been created during the branch to may-aliases (as it is not safe to assume that the branch will be executed in each and every case):

\begin{phpcode}
$a = &$b;
if (...) {
  $c = &$d;
  $e = &$d;
}
\end{phpcode}
$mustAliases = \{(a, b)\}, mayAliases = \{(c, d, e\}$

To ease processing, the alias tuples with more than two elements are split into separate pairs:

$mustAliases = \{(a, b)\}, mayAliases = \{(c, d), (c, e), (d, e)\}$


\section{Alias analysis for the default pass-by-reference in PHP 5}
