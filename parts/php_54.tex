\chapter{PHP 5.4 (WORK IN PROGRESS)}
\label{php54}\index{PHP version 5.4}

\paragraph{Note:} This section will be revamped as a ``changes in different PHP versions'' section in the chapter on PHP.

Pixy in its current version is only able to deal with PHP 4 code. However, in the meantime PHP has progressed to version 5.4. This version has brought some major changes over 4.x that affects static code analysis:

\myTable{
\begin{tabular}{|l|l|}
  \hline
  \bb{New language feature} & \bb{Effect on static code analysis} \\
  \hline
  new keywords & language definition for the lexer/parser\\
  \hline
  constants & the ``place'' abstraction for variables\\
    & (three-address code \emph{P-TAC})\\
  \hline
  default pass-by-reference & alias analysis\\
  \hline
  type hinting & lexer/parser, type inference\\
  \hline
  visibility keywords & lexer/parser,\\
    \emph{private, protected, public} & control-flow analysis, data-flow analysis\\
  \hline
  autoloader & loading of class files\\
  \hline
  namespaces & lexer/parser, loading of class files\\
  \hline
  late static binding & lexer/parser, control-flow graph\\
  \hline
  anonymous functions (from PHP 5.4)& lexer/parser, control-flow analysis\\
  \hline
\end{tabular}
}{Major changes in PHP 5.4 over 4.x}{php54table}

Note: The new visibility keywords affect both the control-flow analysis as well as the data-flow analysis as they influence which methods can be reached from a class at all and which fields are visible.

The following example demonstrates this issue:

\begin{phpcode}
class A {
  /**
   * @var string
   */
  public $publicField = 'public ... ';
  /**
   * @var string
   */
  protected $protectedField = 'protected ... ';
  /**
   * @var string
   */
  private $privateField = 'private ...';
}

class B extends A {
  /**
   * @return string
   */
  public function getFields() {
    return $this->publicField . $this->protectedField .
      $this->privateField;
  }
}

$b = new B();
echo $b->getFields;
\end{phpcode}

This example will echo \texttt{public ... protected ... } as \texttt{\$this->privateField} accesses an (undeclared) field \texttt{B::privateField} (which will have a default value of \texttt{NULL}, which will be automatically cast to an empty string) instead of the existing, but inaccessible \texttt{A::privateField}.
